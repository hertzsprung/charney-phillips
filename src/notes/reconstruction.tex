\documentclass{article}
\usepackage{fullpage}
\usepackage{amsmath}
\usepackage{amssymb}
\usepackage{bm}
\usepackage[hidelinks]{hyperref}
\usepackage{siunitx}

\newcommand{\Co}{C}
\newcommand{\vect}{\mathbf}
\newcommand{\iu}{{i\mkern1mu}}
\newcommand{\iunit}{\boldsymbol{\hat \imath}}
\newcommand{\junit}{\boldsymbol{\hat \jmath}}
\newcommand{\kunit}{\boldsymbol{\hat k}}
\newcommand{\Sf}{\vect{S}_f}
\newcommand{\Mag}[1]{\lvert #1 \rvert}
\newcommand{\magSf}{\Mag{\Sf}}
\newcommand{\gunit}{\vect{\hat{g}}}
\newcommand{\nunit}{\vect{\hat{n}}}
\newcommand{\TODO}[1]{\textcolor{purple}{TODO: \emph{#1}}}

\title{Charney--Phillips cell centre reconstruction}
\author{James Shaw}

\begin{document}
\maketitle

\noindent \texttt{volVectorField x = fvc::reconstruct(surfaceScalarField xf)} is implemented as
\begin{align}
	x = \left( \sum_f \frac{1}{\Mag{\Sf}} \Sf \otimes \Sf \right)^{-1}
	\cdot
	\left( \frac{1}{\Mag{\Sf}} \Sf x_f \right)
\end{align}
where $x$ is defined at cell centres and $x_f$ is defined at face centres.  The surface area vector $\Sf$ has magnitude equal to the area of face $f$ and is directed outward normal to $f$.
Cell centre potential temperature $\theta_C$ is reconstructed from surrounding values of $\theta_f$
\begin{align}
	\theta_C = \gunit \cdot 
	\left\{
		\left( \sum_f \frac{1}{\magSf} \Sf \otimes \Sf \right)^{-1}
		\cdot
		\left( \sum_f \frac{1}{\magSf} \Sf \left( \theta_f \gunit \cdot \nunit \Mag{\Sf} \right) \right)
	\right\}
%
\intertext{where $\gunit$ is the unit vector of gravitational acceleration and the surface normal unit vector $\nunit = \Sf / \Mag{\Sf}$.  Simplifying we get}
%
	\theta_C = \gunit \cdot 
	\left\{
		\left( \sum_f \frac{1}{\magSf} \Sf \otimes \Sf \right)^{-1}
		\cdot
		\left( \sum_f \frac{1}{\magSf} \Sf \left( \theta_f \gunit \cdot \Sf \right) \right)
	\right\}
\end{align}
Notice that $\theta_f \gunit \cdot \Sf / \magSf = 0$ on vertical faces and $\Mag{\theta_f \gunit \cdot \Sf} / \magSf = \theta_f$ on horizontal faces.

Now consider a two-dimensional cell in the $x$--$z$ plane having dimensions $1 \times 1$ such that $\magSf = 1\:\forall\:f$.  The top, right, bottom and left surface area vectors are $\vect{S}_T = \kunit$, $\vect{S}_R = \iunit$, $\vect{S}_B = - \kunit$ and $\vect{S}_L = - \iunit$ respectively.
\begin{align}
	\theta_C &= \gunit \cdot 
	\left\{
		\left( 
			\begin{bmatrix}0 & 0 \\ 0 & 1\times1\end{bmatrix} + 
			\begin{bmatrix}1\times1 & 0 \\ 0 & 0\end{bmatrix} + 
			\begin{bmatrix}0 & 0 \\ 0 & -1\times-1\end{bmatrix} + 
			\begin{bmatrix}-1\times-1 & 0 \\ 0 & 0\end{bmatrix}
		\right)^{-1}
		\cdot
		\left( \theta_B + \theta_T \right) \kunit
	\right\} \\
%
	&= \begin{bmatrix}0 \\ -1\end{bmatrix} \cdot
	\left\{
		\begin{bmatrix}\frac{1}{2} & 0 \\ 0 & \frac{1}{2}\end{bmatrix}
		\cdot
		\begin{bmatrix}0 \\ \theta_B + \theta_T\end{bmatrix}
	\right\} \\
%
	&= \begin{bmatrix}0 \\ \frac{\theta_B + \theta_T}{2} \end{bmatrix}
\end{align}

\end{document}
